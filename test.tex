\documentclass[11pt,]{article}
\usepackage[left=1in,top=1in,right=1in,bottom=1in]{geometry}
\newcommand*{\authorfont}{\fontfamily{phv}\selectfont}
\usepackage{lmodern}


  \usepackage[T1]{fontenc}
  \usepackage[utf8]{inputenc}



\usepackage{abstract}
\renewcommand{\abstractname}{}    % clear the title
\renewcommand{\absnamepos}{empty} % originally center

\renewenvironment{abstract}
 {{%
    \setlength{\leftmargin}{0mm}
    \setlength{\rightmargin}{\leftmargin}%
  }%
  \relax}
 {\endlist}

\makeatletter
\def\@maketitle{%
  \newpage
%  \null
%  \vskip 2em%
%  \begin{center}%
  \let \footnote \thanks
    {\fontsize{18}{20}\selectfont\raggedright  \setlength{\parindent}{0pt} \@title \par}%
}
%\fi
\makeatother




\setcounter{secnumdepth}{0}



\title{Breaking the Power-Sharing Promise: Protests in Post-conflict Society }



\author{\Large Megan Patsch\vspace{0.05in} \newline\normalsize\emph{University of Iowa}  }


\date{}

\usepackage{titlesec}

\titleformat*{\section}{\normalsize\bfseries}
\titleformat*{\subsection}{\normalsize\itshape}
\titleformat*{\subsubsection}{\normalsize\itshape}
\titleformat*{\paragraph}{\normalsize\itshape}
\titleformat*{\subparagraph}{\normalsize\itshape}

\newcommand{\dummy}[1]{#1}

\usepackage{natbib}
\bibpunct{(}{)}{;}{a}{}{,}
\bibliographystyle{apsr}
%\usepackage[strings]{underscore} % protect underscores in most circumstances



\newtheorem{hypothesis}{Hypothesis}
\usepackage{setspace}

\makeatletter
\@ifpackageloaded{hyperref}{}{%
\ifxetex
  \PassOptionsToPackage{hyphens}{url}\usepackage[setpagesize=false, % page size defined by xetex
              unicode=false, % unicode breaks when used with xetex
              xetex]{hyperref}
\else
  \PassOptionsToPackage{hyphens}{url}\usepackage[unicode=true]{hyperref}
\fi
}

\@ifpackageloaded{color}{
    \PassOptionsToPackage{usenames,dvipsnames}{color}
}{%
    \usepackage[usenames,dvipsnames]{color}
}
\makeatother
\hypersetup{breaklinks=true,
%            bookmarks=true,
            pdfauthor={Megan Patsch (University of Iowa)},
             pdfkeywords = {rebels, power-sharing, post-conflict, protest},  
            pdftitle={Breaking the Power-Sharing Promise: Protests in Post-conflict Society},
            colorlinks=true,
            citecolor=black,
            urlcolor=blue,
            linkcolor=black,
            pdfborder={0 0 0}}
\urlstyle{same}  % don't use monospace font for urls

% set default figure placement to htbp
\makeatletter
\def\fps@figure{htbp}
\makeatother



% add tightlist ----------
\providecommand{\tightlist}{%
\setlength{\itemsep}{0pt}\setlength{\parskip}{0pt}}

\begin{document}
	
% \pagenumbering{arabic}% resets `page` counter to 1 
%
% \maketitle

{% \usefont{T1}{pnc}{m}{n}
\setlength{\parindent}{0pt}
\thispagestyle{plain}
{\fontsize{18}{20}\selectfont\raggedright 
\maketitle  % title \par  

}

{
   \vskip 13.5pt\relax \normalsize\fontsize{11}{12} 
\textbf{\authorfont Megan Patsch} \hskip 15pt \emph{\small University of Iowa}   

}

}








\begin{abstract}

    \hbox{\vrule height .2pt width 39.14pc}

    \vskip 8.5pt % \small 

\noindent Increasingly, rebel groups are being incorporated into legitimate
governments as part of liberal peace processes designed to facilitate
inclusion (\citet{hartzell_institutionalizing_2008}). As a result,
former rebels are given political offices in post-conflict governments.
Existing research often treats all rebel groups as similar entities in
both pre-and post-conflict societies. However, research in recent years
has made efforts to establish a more nuanced understanding of these
groups and has highlighted an intriguing trend; rebel groups may provide
public services and create their own bureaucracies in an attempt to gain
support, increase legitimacy, or even decrease the state's legitimacy
during war (see \citet{stewart_civil_2018},
\citet{terpstra_rebel_2017}). These behaviors have unexplored
implications in the post-conflict setting. How do these practices affect
post-conflict governments? I test the effect of power-sharing with
rebels on political protests from 1980 to 2018 and find that political
inclusion of rebels generally as a negative effect on protests. However,
power-sharing with rebels who held wartime elections can increase the
likelihood of protest after war.


\vskip 8.5pt \noindent \emph{Keywords}: rebels, power-sharing, post-conflict, protest \par

    \hbox{\vrule height .2pt width 39.14pc}



\end{abstract}


\vskip 6.5pt


\noindent \singlespacing GovernHist \textless{}- hist(RgdMmFhCorrected\$govern, main=``Histogram
for Governance Institutions'' , xlab=``Number of Institutions'' ,
border=``light blue'' , col=``blue'')

colnames(RgdMmFhCorrected){[}colnames(RgdMmFhCorrected)==``Power-Sharing''{]}
\textless{}- ``Powersharing''
colnames(RgdMmFhCorrected){[}colnames(RgdMmFhCorrected)==``Rebel
Governance Score''{]} \textless{}- ``RebelGovernance''
colnames(RgdMmFhCorrected){[}colnames(RgdMmFhCorrected)==``Government
Victory''{]} \textless{}- ``GovernmentVictory''
colnames(RgdMmFhCorrected){[}colnames(RgdMmFhCorrected)==``Rebel
Victory''{]} \textless{}- ``RebelVictory''
colnames(RgdMmFhCorrected){[}colnames(RgdMmFhCorrected)==``Settlement''{]}
\textless{}- ``Settlement''
colnames(RgdMmFhCorrected){[}colnames(RgdMmFhCorrected)==``coup''{]}
\textless{}- ``Coup''
colnames(RgdMmFhCorrected){[}colnames(RgdMmFhCorrected)==``Civil
Liberties''{]} \textless{}- ``CivilLiberties''
colnames(RgdMmFhCorrected){[}colnames(RgdMmFhCorrected)==``Political
Rights''{]} \textless{}- ``PoliticalRights''

m1 \textless{}- negbinmfx(data = RgdMmFhCorrected, protest1
\textasciitilde{} Powersharing + RebelGovernance + GovernmentVictory +
RebelVictory + Settlement + Coup + CivilLiberties + PoliticalRights + t
+ t2 + t3, clustervar1 = ``region'') summary(m1\(fit) m1\)mfxest

Figure2 \textless{}- as.data.frame(m1\$mfxest) \%\textgreater{}\%
transmute(term = rownames(.), estimate = \texttt{dF/dx}, std.error =
\texttt{Std.\ Err.}) \%\textgreater{}\% dotwhisker::dwplot()

m3 \textless{}- negbinmfx(data = RgdMmFhCorrected, protest1
\textasciitilde{} Powersharing*elect + RebelGovernance +
GovernmentVictory + RebelVictory + Settlement + Coup + CivilLiberties +
PoliticalRights + t + t2 + t3, clustervar1 = ``region'')
summary(m2\(fit) m2\)mfxest

Figure3 \textless{}- as.data.frame(m2\$mfxest) \%\textgreater{}\%
transmute(term = rownames(.), estimate = \texttt{dF/dx}, std.error =
\texttt{Std.\ Err.}) \%\textgreater{}\% dotwhisker::dwplot()

\begin{verbatim}
As civil conflicts approach peaceful resolution, rebel groups are frequently offered power-sharing agreements which utilize democratic principles to alleviate grievances.  Rebel organizations, as a result, are given the opportunity to transition into political life.  Despite the appealingly liberal provisions in power-sharing, these agreements often fail to promote peace [@mukherjee_why_2006].  Success of power-sharing has been measured in the literature in terms of civil war recurrence, but this approach leaves significant uncertainty about lower-level activity in post-war states.  The amount of support that power-sharing governments have, the kind of opposition they face, or even the willingness for former rebels and non-rebel citizens to accept these new governments has been largely unexplored.  
Power-sharing governments rule over not only former rebels and their supporters, but also the wider population of the state.  In some cases, rebel groups have high levels of perceived legitimacy and governing experience from their wartime efforts.  In others, they do not.  Some groups ran extensive healthcare services, collected taxes, and held elections for leadership positions, but others were limited to military goals alone (@terpstra_rebel_2017).  Do these differences affect post-conflict society?  In this paper, I address this question by exploring the relationship between power-sharing governments and anti-government protests.  
\end{verbatim}

\#\#Rebel Groups Research on rebel groups and rebel movements has
historically treated all groups as similar to one another. However,
recent literature has been exploring a more nuanced view of such
organizations (See \citet{huang_wartime_2016},
\citet{stewart_civil_2016}, \citet{terpstra_rebel_2017}). Older research
tended to view rebel groups as highly hierarchical, predominantly
militant, and somewhat one-dimensional. The expectation was that these
groups existed to use violence against the state to achieve a set of
goals. But this is an overly narrow view of such groups, which has
limited our ability to understand their transition into post-conflict
politics. New research has insightfully added dimension and highlighted
important differences between rebel groups which may have critical
implications in the post-war period. One critical source of variation in
rebel groups is wartime governance. Though some rebels are focused on
military victories and tactics, there is a wide range of governance
structures, institutions, and services that other groups provide. Rebel
groups can hold elections, collect taxes, operate police forces, have
parliaments and executive offices, provide educational services, or
provide healthcare facilities in their communities. These services are
sometimes limited to members who support rebel groups, but may be
offered to locals regardless of their support or opposition to rebel
goals (\citet{huang_wartime_2016}). The effects of these wartime efforts
can be huge. The Eritrean People's Liberation Front (EPLF) began as a
guerrilla organization with separatist goals. Over time, however, the
developed an extensive system of public goods provision. Estimates
suggest that by the end of the war the EPLF provided medical care to
more than 1.6 million people. They served an additional 10,000 with
literacy training (\citet{stewart_civil_2018}). Such behavior by rebels
likely has an impact after war.\\
What accounts for this behavior? Though some research suggests that
providing public goods could be a source of recruitment (
\citet{berman_religion_2008}), other research indicates that these
strategies are used towards different goals. Open services in particular
come with no quid-pro-quo wherein citizens are compelled to join the
rebels in exchange for public goods. One possible goal of rebel services
such as these could be legitimacy. When rebel groups are able to provide
security, education, and healthcare to communities it increases views of
their legitimacy by locals. It also helps them to maintain control over
territory, which could be necessary for future conflict or as leverage
when bargaining for peace (\citet{terpstra_rebel_2017}). Using violence,
force, and coercion to control different areas is a resource-heavy drain
on rebel groups. Offering these services, in contrast, is less costly
and more effective in the long run. Increased legitimacy for rebel
groups can also diminish the perceived legitimacy of the official state
government. When rebels are able to provide needed goods and services to
communities, which the state government cannot then they are able to
simultaneously improve their image while chipping away at that of the
``official'' state (\citet{terpstra_rebel_2017}).

\#\#Power-Sharing Power-sharing agreements have been increasingly
popular conflict resolution tools in the last few decades
(\citet{ottmann_power-sharing_201}). This is because they appeal to
liberal ideals such as inclusion, and democracy. If conflict is caused
by political exclusion, grievances, and need then surely the way to
alleviate conflictual societies is to provide a political remedy to
those issues. Yet power-sharing agreements, despite being designed
around such sources of conflict, fail at alarming rates. Mukherjee
(\citet{mukherjee_why_2006}, p.480) conducted one study of 61 cases of
power-sharing and found that in 19 cases (approximately 31\%) conflict
resumed within 6-19 months, in 24 cases (39\%) conflict resumed in 67-94
months, and in 18 cases (30\%) conflict returned in 95-183 months.
Though the duration of peace was longer in some cases than others,
ultimately conflict returned to each case of power-sharing studied by
Mukherjee.\\
Power-sharing agreements can offer rebels different incentives and
offices. As part of an agreement, rebels may be incorporated into the
state, military, or society in new ways. There are four main kinds of
power-sharing provisions: economic, territorial, military, or political.
(\citet{hartzell_institutionalizing_2008}). Often, these categories will
be used in conjunction with one another to address as many grievances as
possible. With economic power-sharing, government-distributed resources
may be allocated or regulated in a way which is more equitable for the
former rebels. Military power-sharing provides an alluring security
incentive for former rebels, but also presents the challenging of
allowing a formerly violent, anti-government groups to retain arms.
Political power-sharing happens in perhaps the most diverse ways. In
some agreements, rebels are allocated specific seats in parliament, or
even executive office. In other cases, former rebels are merely allowed
to run for office and thereby given a chance to legislate (Mukherjee
2006). Agreements which offer multiple forms of power-sharing tend to
experience greater peace after implementation
(\citet{hartzell_institutionalizing_2008}). The democratic appeal of
power-sharing agreements is paramount. Inclusion is inherent in these
deals, regardless of the specific provisions. But another important
consideration is public support. When democratic leaders have high
levels of political trust, they are able to be more successful
(\citet{wang_political_nodate}). It is unclear how much trust
power-sharing governments and political actors are able to garner. As
part of the peacetime arrangements, amnesty for crimes committed during
war is often given to former rebels and their leadership
(\citet{jarstad_war_2008}). This makes the act of reaching an agreement
possible, but presents challenges for those in society who prefer a more
an approach that appeals to those desiring retribution or justice.\\
In 2016, the Revolutionary Armed Forces of Colombia (FARC) were offered
a possible peace deal with the Colombian government after more than 50
years of violent conflict. One provision of this agreement was amnesty
for key members of the organization. As part of the peace process, a
popular vote was used to determine the final agreement. When put to such
a vote, Colombians voted against the deal. One of the main reasons for
this rejection: amnesty (\citet{shifter_will_2016}). A different deal
was later accepted which included harsher punishments for members of
FARC (\citet{katkov_colombias_2016}). Power-sharing with former rebels
who would have been given amnesty was unpopular amongst Colombians. This
vote provides possible insight to society's preferences and perceptions
of rebels. But not all power-sharing agreements are put to a vote; in
many cases once the government declares an agreement completed it is
implemented accordingly. The promise of power-sharing as a tool for
peace is couched in calculated risks. Democracy, accountability, and
representation can be challenged by such agreements. New power-sharing
governments blend former enemies together and ask them to work together
effectively enough that policy continues to be made and implemented. In
these societies there is a high risk of government stalemate, which can
aggravate the same grievances that led to rebel group formation by
failing to produce adequate services and policies. The way that rebels
interacted with society during war can even have an impact on how
democratic the post-conflict government is. If rebels were highly
reliant on civilians during the conflict, particularly near the end of
it, this can have a positive impact on how democratic the post-conflict
government is. However, if rebel institutions remain active after war
ends, the government is likely to be more authoritarian
(\citet{huang_wartime_2016}). Additionally, there is a challenge on the
individual level: the skills which make an effective wartime rebel
leader may not be the same skills that make a strong political leader.
More experience from wartime governance may give some rebels an
advantage in political settings.

\#\#Post-Conflict In the post-conflict era, rebel groups must undergo a
transition to become political parties. These transformations can be
challenging. Groups must transition goals from military aims to policy
proposals, expand their support base to gain votes, and find sources of
funding for their party. The transition to political party from rebel
group can cost valuable support from previous membership, support which
can be challenging to find elsewhere for a party untested at the polls.
Post-conflict elections must also take place, which can be difficult and
even violent. Though post-conflict election are critical for
implementing power-sharing agreements and signaling a return to peaceful
politics, they can have a detrimental effect. In some cases, these
elections may be the source of conflict recurrence
(\citet{keels_oil_nodate} 2015). High levels of political participation
may be a sign of highly mobilized and engaged citizens---a possibly
dangerous ingredient in a society not far removed from war
(\citet{letsa_voting_2017}). However, some research suggests that high
turnout is a sign of tacit support for peace agreements (Kumar 1998).\\
The continued implementation of power-sharing agreements presents an
interesting paradox: if political inclusion is the remedy for
conflictual behavior, why does power-sharing fail at such a high rate?
One of the most understudied aspects of power-sharing is post-conflict
societies. Beyond the likelihood of conflict recurrence, we know little
about how well former rebels can govern or how well they are received as
government officials. Given their previous involvement in violence and
war, it is possible that they are perceived as illegitimate. But wartime
public goods provisions or democratic behavior may change that
perception and increase the likelihood that they are accepted as
political officials. We now know that not all rebel groups are viewed as
illegitimate in wartime by the general population (Terpstra and Frerks
2017). If rebel groups were focused not only on military targets, but
also on community wellbeing, can they be integrated into power-sharing
governments with less friction? One possible measure of support or
opposition to power-sharing governments is protests. Protests are one
possible sign of mobilized and frustrated citizens Protests and Mass
Movements Protest behavior can have mixed implications in a democratic
setting. Collective mobilization is dependent on shared grievances,
social networks, communication ability, and the realm possible actions
available. Tarrow (\citet{tarrow_power_2011}) makes the distinction that
insurgency is characterized by disruption, which may or may not be
associated with mass movements. Movements which rely too heavily on
disruptive behavior are at a higher risk of resorting to violence than
those which adopt alternative ways to address their concerns. One of the
most important factors, according to Tarrow (\citet{tarrow_power_2011}),
is social networks. As part of most peace agreements, rebel groups take
part in demobilization, disarmament, and reintegration (DDR) processes.
These are often led by the state, but may be facilitated by outside
groups involved in conflict resolution with the two sides. DDR is
designed to disarm and demilitarize rebels, making them less of a threat
to society and the government. They also aim to incorporate former
rebels into regular social networks and work environments, though the
extent to which these goals are pursued varies greatly. Research has
shown that even when DDR programs are otherwise successful, rebel
networks remain engaged between individuals who took part in civil wars
(\citet{wiegink_former_2015}). The bonds between former rebels are deep;
often they live in the same places, socialize regularly, and even work
at the same peace-time jobs. These network connections can prove
dangerous if rebel leadership has been coopted by the government, which
is highly possible after power-sharing takes effect (Themner 2015).
Essentially: rebel groups remain largely intact socially, even after
rebel leaders become political and weapons are eliminated. If, as Tarrow
(tarrow\_power\_2011) suggests, social networks are the most necessary
element for mass mobilization then former rebel groups are in a prime
position to be re-engaged.\\
Data regarding public acceptance of power-sharing from non-rebel
supporters is limited, but could provide insight on post-conflict
protests, as well. Few agreements are put to public vote; the Colombia
deal with FARC is one of the most well-known examples. For Colombians,
willingness to accept FARC as a political entity was met with
significant push-back when a referendum vote was first offered in 2016
(\citet{katkov_colombias_2016}). The initial power-sharing agreement was
rejected by popular vote, but the Colombian government and FARC
continued to pursue an inclusive settlement. Even after a different deal
was accepted, the public strongly opposed FARC as political actors.
Colombians began protesting the former rebels as they organized their
new political party and recruited possible candidates. Opposition to
FARC was so strong that they had to suspend campaigns due to threats
towards their candidates in 2018 (BBC February 2018).\\
Post-conflict protests are not inherently sinister. They may signal a
healthy democracy, where citizens are using non-violent behavior to
demonstrate frustration or disagreement with government behavior. Even
protest by the former rebels can be innocuous. Sinn Fein, the political
party which came out of the Irish Republican Army (IRA), started as a
militant organization seeking the reunification of Northern Ireland with
Ireland (Feeny 2003). Sinn Fein ran candidates for local elections long
before the new millennium, but their most impactful political work came
after they were offered seats in the Northern Ireland Assembly through
the Good Friday and St.~Andrews Agreements in 1998 and 2006. After
these, Sinn Fein and their supporters used protests as a means of
opposing policy decisions, and they did so within official government
channels. Whereas activity before power-sharing was conducted without
regard to legal paperwork or policies, Sinn Fein began to file necessary
paperwork for protests once they became a legitimate political party
(Moriarty 2008). In this case, protests were merely another tool in the
new party's political toolbelt.

\#\#Hypotheses and Data Despite potential uncertainty about the exact
meaning behind protests, it seems necessary to first understand whether
power-sharing and rebel wartime experience have an influence at all.
Protests are a sign of negative feelings towards the government, which
is a useful starting point from which we can understand former rebel's
roles in post-conflict governments and society.\\
For this paper I plan to consider two hypotheses:

\begin{quote}
H1: Post-conflict societies with former rebel groups in power-sharing
will experience protests at lower rates than post-conflict societies
without power-sharing.
\end{quote}

\begin{quote}
H2: Power-sharing agreements where former rebels have more wartime
governing experience will have lower levels of protest than those where
former rebels have none.
\end{quote}

The data I am using for this research comes from three main sources.
Data on wartime rebel governance comes from Huang's
(huang\_wartime\_2016) Rebel Governance Dataset which contains
information on the sophistication of rebel group state-building and
governing capacity during war. Protest data comes from the Mass
Mobilization dataset (\citet{clark_mass_2016}), which contains data on
protests in 162 countries. This dataset contains information regarding
location, protest demands, length of event, and state response but for
my purposes I will focus on the number of protests alone. It holds
events from 1990 to 2018. This gives me a dataset with contains eighteen
country-years of protests for countries which experienced civil war
termination between 1989 and 2016. Not all countries exist for the full
18 years, because some were merged into others (i.e.~Yugoslavia). For
this study, I included power-sharing agreements which were implemented
in 1989 as the effect would still be temporally important to the
likelihood of protest. I also include Colombia, though the date of
conflict termination is after the RGD ends. The Colombian war was active
for the time period used in the RGD, so the rebel governance information
is included in the dataset. The protest data was also available, so it
seemed pertinent to keep the case in the data. In total, there are 1,522
observations.\\
My dependent variable is the number of protests in each country for each
year. Protests are a count variable, of anti-government mass
mobilization. Because I am looking for a pattern of protest behavior, my
model includes two or more protests per year as the lowest threshold.
One protest per year may be tied to a specific event or holiday, and I
want to capture ongoing feelings towards power-sharing governments. The
decision to use two protests per year or more is based in the idea that
more than one protest a year has meaning for a state. The choice of two
versus three or four is somewhat arbitrary; I simply chose the number
which seemed to provide the lowest but still significant threshold for
this research. The main independent variable, power-sharing, is a binary
indicator of the presence of a power-sharing arrangement between rebels
and the government. This data comes from Hartzell and Hoddie
(\citet{hartzell_institutionalizing_2008}, 2003) and Mukherjee
(\citet{mukherjee_why_2006}, 2006). Two datasets were used to compile my
power-sharing variable because they cover different timespans and have
some variation on which conflicts were identified. By using both sets, I
was able to get a more expansive list of power-sharing states.\\
Rebel governance is measured as a count variable, where the number of
rebel institutions creates a cumulative score for each group. The term
``institutions'' is used here to describe governance elements including
services, bureaucracy, and political branches. Rebel governance scores
range from 1 to 10, with 10 indicating highest possible number of
institutions measured (\citet{huang_wartime_2016}). For example, a group
who had healthcare services and police forces would have a score of 2.
It should be noted that this variable does not describe how efficiently
these function, or whether their services are restricted to rebel group
members or the broader society. However, the number of institutions is
still a useful, if limited, measure of baseline rebel capacity for
governance and institutional experience. The average score for a rebel
groups is 3.3. Institutions include: taxation, elections, an executive
branch, police, healthcare, education, legal systems (courts), formal
laws, diplomatic exchanges with other countries, and social services
including humanitarian efforts (\citet{huang_wartime_2016}). The
distribution of rebel governance scores can be seen in Figure 1.

\href{C:/Users/Megan/Documents/CMPSPaper/Figure1}{Figure 1: Rebel
Governance Score Distribution}

This model includes binary variables for war termination types:
government victory, rebel victory, or settlements. Because of the unique
properties of coups, they are included as a control as well. These
variables come from the RGD (\citet{huang_wartime_2016}). Government
repression and regime type can influence the likelihood of protest, so I
add a measure to account for freedom in each country. Measures of
political rights and civil liberties are added from Freedom House
(Freedom House 2018). The Freedom House scores are higher when rights
are lower, which can produce unclear results in analysis so for the
purpose of this research I use inverted scores which give higher values
to states with more political rights and civil liberties. To control for
the effect of time, I applied cubic polynomials (Carter and Signorino
2010). This allowed me to easily address the temporal component of my
research.\\
To test these hypotheses, I first ran a negative binomial regression.
The results for Model 1 can be found in Figure 2. In this model,
power-sharing has a negative and statistically significant effect on the
likelihood of protest after war. This means that my first hypothesis is
supported. Power-sharing seems to decrease antigovernment mobilization
in post-war settings. This fits with the expectation implied in
power-sharing and peace agreement research; when agreements allow for
more political inclusion there is less public opposition to the
government.

{[}Figure 2: Negative Binomial Regression Model 1{]}
(C:\Users\Megan\Documents\CMPSPaper\Figure2)

Model 1 also shows that the rebel governance score is positive, but not
statistically significant. Government victory, however, was significant.
Conflicts that were ended with a clear government victory are more
likely to experience protests after war. This likely means that citizens
with grievances were given little when the war ended, which makes
anti-government mobilization a logical conclusion. Civil liberties are
also statistically significant and positive, which means that when
societies have more civil liberties protests are more common. This
result fits with literature about free societies and protests. I then
ran the model with an interaction between rebel groups who held wartime
elections and were incorporated into power-sharing governments. These
can be found in Figure 3. When rebel groups who held public elections
for wartime offices share political power with post-war governments,
there is an increased risk of protest. This is a particularly curious
finding; on one hand, it may indicate that rebels and their supporters
are better adapted to democratic behavior. However, the likelihood of
conflict recurrence after power-sharing is implemented is around 45\%
(\citet{hartzell_institutionalizing_2008}). Protests in this environment
may mean that rebel supporters are still unhappy with the state. Wartime
behavior by rebels can shift perceptions of legitimacy away from the
state and to the group itself. This may have a lasting effect on how the
government is perceived, though the fact that this result is specific to
elections seems to indication a possible democratic behavior component.
Research in the future may consider whether these societies are more or
less likely to return to war. In this model, government victories are
still a positive and significant predictor of protests. Other kinds of
conflict termination did not produce significant results in Model 3.
Civil liberties can also increase the likelihood of protest in
post-conflict society.

\href{C:/Users/Megan/Documents/CMPSPaper/Figure3}{Figure 3: Negative
Binomial Regression Model 2}

\#\#Conclusion When civil conflict ends, societies increasingly rely on
power-sharing agreements to facilitate cooperation with previously
excluded groups in society. However, the effects of these agreements on
post-conflict behavior is unclear. Agreements are designed to promote
democratic ideals such as political inclusion, yet are not always well
received or successful. Rebel and government elites make these deals,
and may lose the support of their bases. Populations may not be willing
to accept former rebels in political roles, and former rebels may be
unwilling to follow leadership into legitimate political life.
Differences in rebel group wartime experience can shed some light on
possible influences to resistance or acceptance of power-sharing
governments. This research shows that power-sharing generally decreases
the likelihood of post-conflict protests. This seems to indicate that
such agreements do what is expected of them; with inclusion comes less
domestic strife. However, not all rebel groups are the same. Rebels who
held wartime elections may present additional challenges to
post-conflict states. Agreements with these rebels is more likely to
generate protests, though it is unclear from this data if it is former
rebels, non-rebel citizens, or both protesting. Future research should
consider whether these societies are more likely to experience conflict
recurrence, to help us understand whether the protests are a sign of
healthy democratic expression or coming turmoil. Electoral participation
may also help to clarify the depth of support for both sides in
power-sharing governments. The rate of power-sharing continues to
increase across the world, making it necessary to study and understand
the ramifications. Providing militant organizations with political power
can be a risky and obstacle-laden path, but it may also be the only way
to truly resolve underlying concerns from excluded members of society.
If power-sharing governments are to be successful going forward, we must
continue to delve into the conditions that they create.

\hypertarget{citations}{%
\subsection{Citations}\label{citations}}

Want to cite something?

\begin{enumerate}
\def\labelenumi{\arabic{enumi}.}
\tightlist
\item
  Find your cite key in your bib file.
\item
  Put an @ before it, like \citet{Solt2017}, or whatever it is
\item
  \citet{Solt2017} creates an in-text citation
\item
  \citep{Herndon2014} creates a parenthetical citation
\end{enumerate}

As \citet{Gelman2014} note, the garden of forking paths can pose
problems for researchers even when they are acting in good faith.

\hypertarget{other-common-things}{%
\subsection{Other Common Things}\label{other-common-things}}

\begin{quote}
This will create a block quote, if you want one. Dropping a footnote is
easy.\footnote{See? Not hard at all.}
\end{quote}




\newpage
\singlespacing 
\bibliography{\dummy {C:/Users/Megan/Documents/CMPSPaper/CMPSRef.bib, C:/Users/Megan/Documents/CMPSPaper/example_article.bib}}

\end{document}